% \pagebreak[4]
% \hspace*{1cm}
% \pagebreak[4]
% \hspace*{1cm}
% \pagebreak[4]

\chapter{Đây là chương 1 }
\ifpdf
    \graphicspath{{Chapter1/Chapter1Figs/PNG/}{Chapter1/Chapter1Figs/PDF/}{Chapter1/Chapter1Figs/}}
\else
    \graphicspath{{Chapter1/Chapter1Figs/EPS/}{Chapter1/Chapter1Figs/}}
\fi

\section{Section đầu tiên}
Viết luận văn bằng  \hologo{LaTeX}. Viết luận văn bằng  \hologo{LaTeX}. Viết luận văn bằng  \hologo{LaTeX}. Viết luận văn bằng  \hologo{LaTeX}. Viết luận văn bằng  \hologo{LaTeX}. Viết luận văn bằng  \hologo{LaTeX}. Viết luận văn bằng  \hologo{LaTeX}. Viết luận văn bằng  \hologo{LaTeX}. Viết luận văn bằng  \hologo{LaTeX}. Viết luận văn bằng  \hologo{LaTeX}. Viết luận văn bằng  \hologo{LaTeX}. Viết luận văn bằng  \hologo{LaTeX}. Viết luận văn bằng  \hologo{LaTeX}. Viết luận văn bằng  \hologo{LaTeX}. Viết luận văn bằng  \hologo{LaTeX}. Viết luận văn bằng  \hologo{LaTeX}. Viết luận văn bằng  \hologo{LaTeX}. Viết luận văn bằng  \hologo{LaTeX}. Viết luận văn bằng  \hologo{LaTeX}. Viết luận văn bằng  \hologo{LaTeX}. Viết luận văn bằng  \hologo{LaTeX}.  

Thử nghiệm trích dẫn \cite{Andress2002WLANSecurity}. Trích dẫn tên tác giả \citet{Barbeau2012P2PvoiceOverAdHocSurvey} 
Trích dẫn toàn bộ các tác giả \citet*{Bennett2012MobileDeviceForensicsChallenges}. 

Viết luận văn bằng  \hologo{LaTeX}. Viết luận văn bằng  \hologo{LaTeX}. Viết luận văn bằng  \hologo{LaTeX}. Viết luận văn bằng  \hologo{LaTeX}. Viết luận văn bằng  \hologo{LaTeX}. Viết luận văn bằng  \hologo{LaTeX}. Viết luận văn bằng  \hologo{LaTeX}. Viết luận văn bằng  \hologo{LaTeX}. Viết luận văn bằng  \hologo{LaTeX}. Viết luận văn bằng  \hologo{LaTeX}. Viết luận văn bằng  \hologo{LaTeX}. Viết luận văn bằng  \hologo{LaTeX}. Viết luận văn bằng  \hologo{LaTeX}. Viết luận văn bằng  \hologo{LaTeX}. Viết luận văn bằng  \hologo{LaTeX}. Viết luận văn bằng  \hologo{LaTeX}. Viết luận văn bằng  \hologo{LaTeX}. Viết luận văn bằng  \hologo{LaTeX}. Viết luận văn bằng  \hologo{LaTeX}. Viết luận văn bằng  \hologo{LaTeX}. Viết luận văn bằng  \hologo{LaTeX}. Viết luận văn bằng  \hologo{LaTeX}. Viết luận văn bằng  \hologo{LaTeX}. Viết luận văn bằng  \hologo{LaTeX}. Viết luận văn bằng  \hologo{LaTeX}. Viết luận văn bằng  \hologo{LaTeX}. Viết luận văn bằng  \hologo{LaTeX}. Viết luận văn bằng  \hologo{LaTeX}. Viết luận văn bằng  \hologo{LaTeX}. Viết luận văn bằng  \hologo{LaTeX}. 
Thử nghiệm trích dẫn \citep{Bertino2009LAAandACC}. Trích dẫn tên tác giả \citet{Breu2004TowardsASystemmaticDevelopmentOfSecureSystems} 
Trích dẫn toàn bộ các tác giả \citet*{bunting2012encase}. 


Viết luận văn bằng  \hologo{LaTeX}. Viết luận văn bằng  \hologo{LaTeX}. Viết luận văn bằng  \hologo{LaTeX}. Viết luận văn bằng  \hologo{LaTeX}. Viết luận văn bằng  \hologo{LaTeX}. Viết luận văn bằng  \hologo{LaTeX}. Viết luận văn bằng  \hologo{LaTeX}. Viết luận văn bằng  \hologo{LaTeX}. Viết luận văn bằng  \hologo{LaTeX}. Viết luận văn bằng  \hologo{LaTeX}. Viết luận văn bằng  \hologo{LaTeX}. Viết luận văn bằng  \hologo{LaTeX}. Viết luận văn bằng  \hologo{LaTeX}. Viết luận văn bằng  \hologo{LaTeX}. Viết luận văn bằng  \hologo{LaTeX}. Viết luận văn bằng  \hologo{LaTeX}. Viết luận văn bằng  \hologo{LaTeX}. Viết luận văn bằng  \hologo{LaTeX}. Viết luận văn bằng  \hologo{LaTeX}. Viết luận văn bằng  \hologo{LaTeX}. Viết luận văn bằng  \hologo{LaTeX}. Viết luận văn bằng  \hologo{LaTeX}. Viết luận văn bằng  \hologo{LaTeX}. Viết luận văn bằng  \hologo{LaTeX}. Viết luận văn bằng  \hologo{LaTeX}. Viết luận văn bằng  \hologo{LaTeX}. Viết luận văn bằng  \hologo{LaTeX}. Viết luận văn bằng  \hologo{LaTeX}. Viết luận văn bằng  \hologo{LaTeX}. Viết luận văn bằng  \hologo{LaTeX}. Viết luận văn bằng  \hologo{LaTeX}. Viết luận văn bằng  \hologo{LaTeX}. Viết luận văn bằng  \hologo{LaTeX}. 


Viết luận văn bằng  \hologo{LaTeX}. Viết luận văn bằng  \hologo{LaTeX}. Viết luận văn bằng  \hologo{LaTeX}. Viết luận văn bằng  \hologo{LaTeX}. Viết luận văn bằng  \hologo{LaTeX}. Viết luận văn bằng  \hologo{LaTeX}. Viết luận văn bằng  \hologo{LaTeX}. Viết luận văn bằng  \hologo{LaTeX}. Viết luận văn bằng  \hologo{LaTeX}. Viết luận văn bằng  \hologo{LaTeX}. Viết luận văn bằng  \hologo{LaTeX}. Viết luận văn bằng  \hologo{LaTeX}. Viết luận văn bằng  \hologo{LaTeX}. Viết luận văn bằng  \hologo{LaTeX}. Viết luận văn bằng  \hologo{LaTeX}. Viết luận văn bằng  \hologo{LaTeX}. Viết luận văn bằng  \hologo{LaTeX}. Viết luận văn bằng  \hologo{LaTeX}. Viết luận văn bằng  \hologo{LaTeX}. Viết luận văn bằng  \hologo{LaTeX}. Viết luận văn bằng  \hologo{LaTeX}. 


Thử nghiệm trích dẫn \cite{Ahamed2010FTMforPervasiveEnvironments}. Trích dẫn tên tác giả \citet{Ahamed2010FTMforPervasiveEnvironments}. Trích dẫn đầy đủ 
\citet*{Ahamed2010FTMforPervasiveEnvironments}. 

Để học thêm về cách sử dụng gói trích dẫn natbib, các bạn cần đọc thêm tại đây: http://casa.colorado.edu/\~danforth/comp/tex/tutorial.html

Here is an equation\footnote{the notation is explained in the nomenclature section :-)}:
\begin{eqnarray}
CIF: \hspace*{5mm}F_0^j(a) &=& \frac{1}{2\pi \iota} \oint_{\gamma} \frac{F_0^j(z)}{z - a} dz
\end{eqnarray}
\nomenclature[zcif]{$CIF$}{Cauchy's Integral Formula}                                % first letter Z is for Acronyms 
\nomenclature[aF]{$F$}{complex function}                                                   % first letter A is for Roman symbols
\nomenclature[gp]{$\pi$}{ $\simeq 3.14\ldots$}                                             % first letter G is for Greek Symbols
\nomenclature[gi]{$\iota$}{unit imaginary number $\sqrt{-1}$}                      % first letter G is for Greek Symbols
\nomenclature[gg]{$\gamma$}{a simply closed curve on a complex plane}  % first letter G is for Greek Symbols
\nomenclature[xi]{$\oint_\gamma$}{integration around a curve $\gamma$} % first letter X is for Other Symbols
\nomenclature[rj]{$j$}{superscript index}                                                       % first letter R is for superscripts
\nomenclature[s0]{$0$}{subscript index}                                                        % first letter S is for subscripts

\section{Section  tiếp theo}
 

\subsection{Đây là subsection }
\subsubsection{Đây là sunsubsection }
 
Hạn chế dùng đến x.x.x.... 
\subsection{Đây là subsection tiếp theo}
... and some more ...

Now I would like to cite the following: 
and \cite{Rud73}.

I would also like to include a picture ...

\begin{figure}[!htbp]
  \begin{center}
    \leavevmode
    \ifpdf
      \includegraphics[height=6in]{aflow}
    \else
      \includegraphics[bb = 92 86 545 742, height=6in]{aflow}
    \fi
    \caption{Airfoil Picture}
    \label{FigAir}
  \end{center}
\end{figure}

% above code has been macro-fied in Classes/MacroFile.tex file
%\InsertFig{\IncludeGraphicsH{aflow}{6in}{92 86 545 742}}{Airfoil Picture}{FigAir}

So as we have now labelled it we can reference it, like so (\ref{FigAir}) and it
is on Page \pageref{FigAir}. And as we can see, it is a very nice picture and we
can talk about it all we want and when we are tired we can move on to the next
chapter ...

I would also like to add an extra bookmark in acroread like so ...
\ifpdf
  \pdfbookmark[2]{bookmark text is here}{And this is what I want bookmarked}
\fi


\section{Kết chương}
Viết luận văn bằng  \hologo{LaTeX}. Viết luận văn bằng  \hologo{LaTeX}. Viết luận văn bằng  \hologo{LaTeX}. Viết luận văn bằng  \hologo{LaTeX}. Viết luận văn bằng  \hologo{LaTeX}. Viết luận văn bằng  \hologo{LaTeX}. Viết luận văn bằng  \hologo{LaTeX}. Viết luận văn bằng  \hologo{LaTeX}. Viết luận văn bằng  \hologo{LaTeX}. Viết luận văn bằng  \hologo{LaTeX}. Viết luận văn bằng  \hologo{LaTeX}. Viết luận văn bằng  \hologo{LaTeX}. Viết luận văn bằng  \hologo{LaTeX}. Viết luận văn bằng  \hologo{LaTeX}. Viết luận văn bằng  \hologo{LaTeX}. Viết luận văn bằng  \hologo{LaTeX}. Viết luận văn bằng  \hologo{LaTeX}. Viết luận văn bằng  \hologo{LaTeX}. Viết luận văn bằng  \hologo{LaTeX}. Viết luận văn bằng  \hologo{LaTeX}. Viết luận văn bằng  \hologo{LaTeX}.  

